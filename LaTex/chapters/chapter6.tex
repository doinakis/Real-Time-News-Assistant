% Conclusions
\chapter{Συμπεράσματα}
\label{chapter:conclusions}
Στο κεφάλαιο αυτό παρουσιάζονται τα συμπεράσματα που προέκυψαν από το σύνολο των πειραμάτων του συνολικού συστήματος. Επιπλέον, γίνεται αναφορά στα προβλήματα που παρουσιάστηκαν τόσο κατά τη διάρκεια των πειραμάτων όσο και κατά τη διάρκεια των υλοποιήσεων.

\section{Γενικά Συμπεράσματα}
Αρχικά, το \emph{RASA} είναι ένα χρήσιμο εργαλείο που επιτρέπει την ανάπτυξη βοηθών σε οποιαδήποτε γλώσσα. Το αρχικό σύνολο εκπαίδευσης του βοηθού δεν χρειάζεται να είναι αρκετά μεγάλο και μπορεί να βελτιώνεται στην πορεία από την προσθήκη νέων συνομιλιών. Αυτό που αξίζει να σημειωθεί είναι ότι η ανάπτυξη του βοηθού αποτελεί μία συνεχή και αδιάκοπη διαδικασία. Το μοντέλο του βοηθού πρέπει να ανανεώνεται συχνά από τα δεδομένα των χρηστών, έτσι ώστε να είναι σε θέση να καλύπτει μεγάλο εύρος εισόδων.

Επιπλέον, όσον αφορά την ταξινόμηση των άρθρων, μελετήθηκαν δύο υλοποιήσεις: η κατασκευή ενός \emph{MLP} και το \emph{fine-tuning} του μοντέλου \emph{Greek-Bert}. Το μοντέλο \emph{MLP}, λόγω της απλότητας του, ήταν σαφώς πιο γρήγορο, τόσο στην εκπαίδευση όσο και στην πρόβλεψη της κατηγορίας του άρθρου, συγκριτικά με το μοντέλο \emph{Greek-BERT}. Ωστόσο, το συγκεκριμένο μοντέλο χρησιμοποιεί τη μέθοδο \emph{TF-IDF} για την εξαγωγή των χαρακτηριστικών των εισερχόμενων άρθρων. Επομένως, δεν συνδέει νοηματικά το άρθρο με την κατηγορία. Για παράδειγμα, σε περίπτωση που εμφανιστεί το όνομα μιας νέας ταινίας στη ταξινόμηση ενός καινούργιου άρθρου, τότε είναι πιθανό να ταξινομηθεί λανθασμένα. Το μοντέλο \emph{Greek-BERT} από την άλλη, με τη χρήση \emph{transformers}, δίνει σημασία τόσο στη σειρά των λέξεων όσο και στη νοηματική τους σύνδεση. Για το λόγο αυτό επιλέχθηκε το μοντέλο \emph{Greek-BERT}.

Επίσης, η συνολική απόδοση του συστήματος \emph{QA} εξαρτάται από το μοντέλο με τη μικρότερη απόδοση. Για παράδειγμα αν ο \emph{Retriever} έχει απόδοση $98\%$, το συνολικό σύστημα δεν μπορεί να έχει απόδοση μεγαλύτερη από αυτή. Επομένως, τα \emph{components} που προστίθενται και βρίσκονται πριν από τον \emph{Reader} πρέπει να έχουν όσο το δυνατό μεγαλύτερη απόδοση, προκειμένου να φτάσει σε αυτόν το έγγραφο που περιέχει τη σωστή απάντηση.

Από τον πειραματισμό με διάφορα μοντέλα για τον \emph{Reader} αποδείχθηκε ότι τα πολυ-γλωσσικά μοντέλα δύναται να έχουν αρκετά καλή απόδοση, ακόμη και σε σχέση με αντίστοιχα μονο-γλωσσικά. Έτσι, αποδεικνύεται ότι η αυτοματοποιημένη εύρεση απάντησης μπορεί να πραγματοποιηθεί. Ωστόσο, η μελέτη και η ανάπτυξη μονο-γλωσσικών μοντέλων είναι απαραίτητη καθώς υπάρχει ακόμη αρκετό περιθώριο βελτίωσης στα μοντέλα ερώτησης-απάντησης.

\section{Προβλήματα}
Ένα από τα κυριότερα προβλήματα που παρουσιάστηκαν κατά τον σχεδιασμό και την υλοποίηση του συστήματος ήταν η έλλειψη μοντέλων ερώτησης-απάντησης στην ελληνική γλώσσα. Αξίζει να σημειωθεί ότι τα ελληνικά μοντέλα που παρουσιάζονται αναπτύχθηκαν ταυτόχρονα με την παρούσα αναφορά, στα πλαίσια διαφορετικής διπλωματικής εργασίας \cite{Nastos2022}.

Ένα ακόμη πρόβλημα παρουσιάστηκε στην προσομοίωση του εξωτερικού συστήματος παροχής των άρθρων. Πιο συγκεκριμένα, όπως αναφέρθηκε στο \autoref{sec:Simulation}, η λήψη των άρθρων πραγματοποιείται με τη μέθοδο \emph{web-scrapping}. Η μέθοδος αυτή, είναι βασισμένη στην αναζήτηση συγκεκριμένων στοιχείων της ιστοσελίδας για την εξαγωγή τόσο του άρθρου όσο και του τίτλου του. Ωστόσο, διαφορετικές ιστοσελίδες έχουν διαφορετική δομή, με αποτέλεσμα να αλλάζει η διαδικασία επεξεργασίας της λαμβανόμενης ιστοσελίδας, αναλόγως με το \emph{site} από το οποίο προήλθε. Ταυτόχρονα, αν κάποια από τις υποστηριζόμενες ιστοσελίδες αλλάξει δομή, επακόλουθα είναι απαραίτητο να αλλάξει και η διαδικασίας λήψης των άρθρων για τη συγκεκριμένη σελίδα. Επομένως, καθίσταται αναγκαία η υλοποίηση ενός σταθερού συστήματος λήψης άρθρων που δεν θα εξαρτάται από αυτές τις παραμέτρους.