% Problem Description
\section{Περιγραφή του προβλήματος}
Ένας από τους τομείς που υποφέρει από καθημερινό "βομβαρδισμό" πληροφορίας είναι αυτός των ειδήσεων. Εκατοντάδες χιλιάδες άρθρα δημοσιεύονται καθημερινά καθιστώντας αδύνατη την ανάγνωση και την εξαγωγή πληροφορίας από αυτά. Επιπλέον, ένα άρθρο είναι σχετικό με την επικαιρότητα μόνο για περιορισμένο χρονικό διάστημα, καθώς πολλές από τις ειδήσεις αλλάζουν λεπτό προς λεπτό. Συνεπώς, η εξαγωγή της πληροφορίας πρέπει να γίνεται σε πραγματικό χρόνο (\emph{real time}) ώστε η είδηση να είναι έγκυρη. Δεν υπάρχουν πολλά συστήματα \emph{QA} που να ανιχνεύουν την απάντηση μιας ερώτησης ανάμεσα σε μία συλλογή από έγγραφα, τα οποία συλλέγονται σε πραγματικό χρόνο. Η συνεχής εξέλιξη των ψηφιακών βοηθών φαίνεται να στοχεύει στην επίλυση του προβλήματος αυτού.

Δυστυχώς, η έλλειψη αυτών των συστημάτων συνεχίζεται και στα ελληνικά. Τόσο συστήματα \emph{QA} αλλά και ψηφιακών βοηθών είναι λίγα σε αριθμό. Η απουσία τέτοιου είδους συστημάτων στην ελληνική οφείλεται κυρίως στην έλλειψη ικανοποιητικού συνόλου δεδομένων για την εκπαίδευση μοντέλων μηχανικής μάθησης αλλά και στο γεγονός ότι τα ελληνικά δυσχεραίνουν το πρόβλημα κατανόησης φυσικής γλώσσας εξαιτίας των συντακτικών και κλιτικών ιδιαιτεροτήτων τους.