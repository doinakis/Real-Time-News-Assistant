% Purpose and Contribution of Thesis
\section{Σκοπός της διπλωματικής εργασίας}
Η παρούσα διπλωματική εργασία μελετά την ανάπτυξη και συντήρηση ενός ειδησεογραφικού βοηθού πραγματικού χρόνου, ο οποίος θα αναγνωρίζει τις επιθυμίες του χρήστη και θα παρουσιάζει σε αυτόν απαντήσεις με εύγλωττο τρόπο. Οι χρήστες θα έχουν τη δυνατότητα να διατυπώνουν ερωτήσεις για την επικαιρότητα οι οποίες θα αφορούν ειδήσεις που πραγματεύονται \textbf{πολιτικά}, \textbf{αθλητικά}, \textbf{τεχνολογία}, \textbf{ταινίες} και \textbf{ηλεκτρονικά παιχνίδια} (\textbf{gaming}).

Επιπλέον, μελετάται η ανάπτυξη ενός συστήματος ερώτησης-απάντησης. Το σύστημα αυτό θα επικοινωνεί με το σύστημα του ψηφιακού βοηθού για τη λήψη της ερώτησης και την επιστροφή της προβλεπόμενης απάντησης. Παράλληλα, για την ανάπτυξη και τη συνεχή βελτίωση του ψηφιακού βοηθού στις εισόδους των χρηστών έχει αναπτυχθεί ψηφιακή διεπαφή μέσω της οποίας γίνεται και η αξιολόγησή του. Ακόμη, μελετάται και η εκπαίδευση μοντέλων κατανόησης φυσικής γλώσσας με στόχο την ταξινόμηση άρθρων στις παραπάνω κατηγορίες.

Με τον τρόπο αυτό προκύπτει ένα συνολικό σύστημα το οποίο αποτελείται από τρία βασικά μέρη:
\begin{itemize}
    \item Ψηφιακός βοηθός
    \item Σύστημα \emph{QA}
    \item Ταξινομητής άρθρων
\end{itemize}
Το πρώτο μέρος είναι ο ψηφιακός βοηθός με τον οποίο αλληλεπιδρά άμεσα ο χρήστης και είναι υπεύθυνος για την κατανόηση των επιθυμιών του χρήστη και την παρουσίαση των απαντήσεων. Το δεύτερο μέρος είναι το σύστημα το οποίο είναι υπεύθυνο για την επεξεργασία του ερωτήματος του χρήστη και την εξαγωγή της σωστής απάντησης. Τέλος, το τρίτο μέρος του συστήματος είναι ο ταξινομητής ο οποίος είναι υπεύθυνος για την ταξινόμηση των εισερχόμενων στο σύστημα άρθρων. Έτσι, δημιουργείται ένα αρθρωτό (\emph{modular}) σύστημα του οποίου τα κομμάτια θα μπορούν στο μέλλον να βελτιωθούν ή και να αντικατασταθούν ολοκληρωτικά προκειμένου να βελτιωθεί η απόδοση του συνολικού συστήματος.