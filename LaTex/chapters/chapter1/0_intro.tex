% General Intro
Τα τελευταία χρόνια, με τη ραγδαία εξέλιξη της τεχνολογίας, έχει σημειωθεί σημαντική αύξηση του όγκου δεδομένων που συσσωρεύονται στο διαδίκτυο. Η παραγωγή πληροφορίας αυξάνεται με εκθετικούς ρυθμούς, με το 2020 το περιεχόμενο του διαδικτύου να φτάνει τα 40 \emph{zetabytes} \footnote{\url{https://healthit.com.au/how-big-is-the-internet-and-how-do-we-measure-it/}}. Στην προσπάθεια διαχείρισης αυτού του τεράστιου όγκου δεδομένων έχουν δημιουργηθεί συστήματα τα οποία διευκολύνουν την αναζήτηση και την ταξινόμηση της ογκώδους αυτής πληροφορίας. Η κατανόηση και η  επεξεργασία της φυσικής γλώσσας (\emph{Natural Language Understanding - NLU} και \emph{Natural Language Processing - NLP}) αποτελούν κλάδους της μηχανικής μάθησης (\emph{Machine Learning - ML}), που έχουν ως στόχο την κατανόηση της σε νοηματικό επίπεδο, και την επεξεργασία της για την εξαγωγή και παραγωγή πληροφορίας.

Αρκετά χρόνια τώρα η επιστήμη αναζητούσε αποδοτικούς τρόπους για την αναζήτηση και την ανάκτηση πληροφορίας (\emph{Information Retrieval - IR}), από την ήδη υπάρχουσα αποθηκευμένη γνώση, χωρίς να είναι απαραίτητη ανθρώπινη παρέμβαση. Πιο συγκεκριμένα, την ύπαρξη ενός συστήματος που θα επέστρεφε στο χρήστη την επιθυμητή πληροφορία χωρίς ο ίδιος να καταβάλει προσπάθεια. Η ιδέα της αυτοματοποιημένης ανάκτησης πληροφορίας παρουσιάστηκε πρώτη φορά το 1945 από τον \emph{Vannevar Bush} στο άρθρο του \emph{As We May Think} \cite{bush1945} με τα πρώτα συστήματα να κάνουν την εμφάνιση τους το 1950. Φτάνοντας στο σήμερα οι μηχανές αναζήτησης του διαδικτύου επιτυγχάνουν το σκοπό αυτό, αναζήτηση σε μεγάλες βάσεις δεδομένων και την επιστροφή των πιο κοντινών, στην είσοδο του συστήματος, εγγράφων. Στη συνέχεια η αναζήτηση συνεχίζεται από τον χρήστη στον περιορισμένο αριθμό εγγράφων που επέστρεψε η μηχανή αναζήτησης.

Το επόμενο βήμα που φαίνεται να φέρνει την επανάσταση στον τρόπο ανάκτησης πληροφορίας είναι τα συστήματα ερώτησης-απάντησης (\emph{Question-Answering - QA}). Για τον σκοπό απαιτείται συνδυασμός των αρχών του \emph{IR} και των \emph{NLU-NLP}. Ως είσοδος στα παραπάνω αποτελούν ερωτήσεις πάνω σε κάποιον τομέα, και ως έξοδος του συστήματος δεν είναι πλέον ένα σύνολο εγγράφων, αλλά συγκεκριμένη πρόταση, η οποία αποτελεί απάντηση στο ερώτημα. Τα συστήματα αυτά μπορούν να χωριστούν σε δύο κατηγορίες: 
\begin{itemize}
    \item \textbf{ανοιχτού πεδίου}: το σύστημα είναι σε θέση να απαντήσει οποιαδήποτε ερώτηση
    \item \textbf{κλειστού πεδίου}: το σύστημα παρέχει απαντήσεις σε συγκεκριμένους τομείς, για παράδειγμα στον τομέα της ιατρικής, φαρμακευτικής, ειδήσεων και άλλων.
\end{itemize}