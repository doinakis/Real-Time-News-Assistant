\section{RASA framework}
\label{sec:rasa}
Το \emph{RASA}\footnote{\url{https://rasa.com/docs/rasa/}} είναι ένα εργαλείο ανοιχτού λογισμικού το οποίο χρησιμοποιείται για την ανάπτυξη ψηφιακών βοηθών. Πιο συγκεκριμένα, είναι υπεύθυνο για την κατανόηση κειμένου (συνήθως συζητήσεων) και την αυτοματοποιημένη παραγωγή κειμένου και φωνής. Η κατανόηση του κειμένου πραγματοποιείται μέσω της αναγνώρισης επιθυμιών (\emph{intents}) αλλά και οντοτήτων (\emph{entities}) στο κείμενο εισόδου. Τα \emph{intents} είναι οτιδήποτε προσπαθεί ο εκάστοτε χρήστης του συστήματος να επιτύχει, για παράδειγμα έναν χαιρετισμό, μία ερώτηση, τον προσδιορισμό μιας τοποθεσίας και άλλα. Τα \emph{entities} είναι λέξεις-κλειδιά που εξάγονται από το κείμενο και ενδέχεται να δείχνουν αριθμούς τηλεφώνου, το όνομα κάποιου, μία τοποθεσία, χρηματικά ποσά και άλλα. 

Από τα πιο σημαντικά κομμάτια ενός ψηφιακού βοηθού είναι να προβαίνει σε κάποια ενέργεια (\emph{action}) ανάλογα με την επιθυμία του χρήστη. Αυτή ακριβώς τη λειτουργικότητα παρέχει ο \emph{action server}. Για κάθε αίτημα, ο \emph{server}, έχοντας την πληροφορία της επιθυμίας και των εξαγόμενων οντοτήτων, εκτελεί το αντίστοιχο κομμάτι κώδικα και επιστρέφει την απάντηση στον ψηφιακό βοηθό. Η επικοινωνία του ψηφιακού βοηθού με τον \emph{server} πραγματοποιείται μέσω REpresentational State Transfer Application Programming Interface - \emph{REST APIs}\footnote{\url{https://restfulapi.net/}}.

Η μεγαλύτερη πρόκληση στη δημιουργία και τη συντήρηση των ψηφιακών βοηθών αποτελεί το γεγονός ότι είναι απίθανη η πρόβλεψη όλων των εισόδων του χρήστη στο σύστημα. Σε κάθε συνομιλία οι χρήστες διατυπώνουν ακριβώς αυτό που επιθυμούν. Ο βοηθός εκπαιδεύεται σε ένα σύνολο προκαθορισμένων συνομιλιών, στις οποίες καλείται να ανταποκριθεί. Οι συνομιλίες αυτές ορίζουν το σύνολο των χαρούμενων μονοπατιών (\emph{happy paths}), δηλαδή συνομιλίες που ο ψηφιακός βοηθός αναμένεται να διαχειριστεί. Έτσι, αφού εκπαιδευτεί ένας ψηφιακός βοηθός που είναι θέση να χειριστεί τα περισσότερα \emph{happy paths}, τότε χρησιμοποιείται το \emph{RASA-X}\footnote{\url{https://rasa.com/docs/rasa-x/}} για τη βελτίωση του βοηθού. Ωστόσο, μία συνομιλία ενδέχεται να ακολουθήσει διαφορετικά βήματα από αυτά που αναμένει ο ψηφιακός βοηθός, τα οποία ονομάζονται \emph{un-happy paths}. Το \emph{RASA-X} είναι ένα εργαλείο κλειστού λογισμικού το οποίο χρησιμοποιείται για ανάπτυξη βοηθών τροφοδοτούμενη από συνομιλίες (\emph{Conversation-Driven Development}), δηλαδή της διαδικασίας παρακολούθησης των συζητήσεων των χρηστών, την αξιολόγηση και την αξιοποίηση τους για την βελτιστοποίηση του ψηφιακού βοηθού. 

