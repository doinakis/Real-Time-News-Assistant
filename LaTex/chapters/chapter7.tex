% Future Work
\chapter{Μελλοντικές επεκτάσεις}
\label{chapter:future_work}
Στα πλαίσια της παρούσας αναφοράς, παρουσιάστηκε ένα ολοκληρωμένο σύστημα ψηφιακού βοηθού ο οποίος θα είναι σε θέση να απαντάει στις ερωτήσεις των χρηστών του που αφορούν θέματα ειδήσεων. Ως αρχική μελλοντική επέκταση, μπορεί να μελετηθεί το σύστημα παροχής άρθρων στο συνολικό σύστημα. Η κύρια πρόκληση που καλείται να αντιμετωπίσει ένα τέτοιο σύστημα είναι η ανάκτηση άρθρων από μεγάλο αριθμό ιστοσελίδων. Πιο συγκεκριμένα, είναι θεμιτό και αναγκαίο τα άρθρα να προέρχονται από πολλές ιστοσελίδες ώστε να αποφεύγεται η μονόπλευρη ενημέρωση των χρηστών. Μία επιπλέον χρήσιμη λειτουργία θα ήταν να μπορούν οι χρήστες να επιλέγουν τις πηγές των ειδήσεων τους.

Μία ακόμη επέκταση του συστήματος είναι η ταξινόμηση των εισερχόμενων ερωτήσεων στο σύστημα \emph{QA}. Πιο συγκεκριμένα, η κατηγοριοποίηση των ερωτήσεων, δεδομένου ότι οι οι ερωτήσεις ταξινομούνται στις σωστές κατηγορίες με υψηλή ακρίβεια, θα μεγιστοποιήσει την πιθανότητα εύρεσης του σωστού άρθρου από το \emph{Retriever}, καθώς η αναζήτηση θα περιορίζεται μόνο στα άρθρα της "σωστής" κατηγορίας. Για την επίτευξη του στόχου αυτού, προτείνεται η συλλογή ερωτήσεων από δοκιμαστικό κοινό (\emph{crowdsourcing}), για κάθε κατηγορία, και το \emph{fine-tuning} του ελληνικού \emph{BERT} μοντέλου όπως παρουσιάστηκε και για την ταξινόμηση των άρθρων. Για τη συλλογή των ερωτήσεων μπορεί να χρησιμοποιηθεί το εργαλείο \emph{RASA-X}.

Επιπλέον, είναι σημαντική η περαιτέρω αξιολόγηση του τρέχοντος συστήματος σε πραγματικές συνθήκες. Η έκθεση του βοηθού σε δοκιμαστικό κοινό θα βοηθήσει στη συλλογή πραγματικών συνομιλιών οι οποίες στη συνέχεια μπορούν να αξιολογηθούν και να χρησιμοποιηθούν για \emph{Conversation Driven Developent}. Με τον τρόπο αυτό, είναι δυνατή η κατανόηση των αδυναμιών του βοηθού αλλά και η αξιοποίηση του εργαλείου \emph{RASA-X}, μέσω του οποίου θα βελτιώνεται συνεχώς. Στο σημείο αυτό αξίζει να σημειωθεί ότι η συντήρηση του ψηφιακού βοηθού αποτελεί μία συνεχή διαδικασία, η οποία πρέπει να πραγματοποιείται και να επιβλέπεται τακτικά, καθώς με τη πάροδο του χρόνου οι συμπεριφορές των χρηστών αλλάζουν, εξελίσσονται και το ίδιο πρέπει να συμβαίνει και με τον ψηφιακό βοηθό.

Ένα από τα κύρια πλεονεκτήματα του συστήματος είναι η ευελιξία που παρέχει. Ο ψηφιακός βοηθός μπορεί να ενσωματωθεί σε γνωστές εφαρμογές επικοινωνίας όπως το \emph{messenger}\footnote{\url{https://www.messenger.com/}}, το \emph{mattermost}\footnote{\url{https://mattermost.com/}}, το \emph{slack}\footnote{\url{https://slack.com/}}, το \emph{discord}\footnote{\url{https://discord.com/}} και άλλες. Με τον τρόπο αυτό το σύστημα θα είναι προσβάσιμο από τους χρήστες σε περιβάλλον με το οποίο είναι ήδη εξοικειωμένοι.

Τέλος, εξαιτίας της αρθρωτής δομής του συνολικού συστήματος, οποιοδήποτε κομμάτι του μπορεί στο μέλλον να αντικατασταθεί για την βελτίωση του. Για παράδειγμα, τόσο το μοντέλο που επιλέχθηκε για τον \emph{Retriever}, όσο και το μοντέλο που επιλέχθηκε για τον \emph{Reader} και το μοντέλο ταξινόμησης μπορούν να αντικατασταθούν από αποδοτικότερα που θα αναπτυχθούν μελλοντικά.