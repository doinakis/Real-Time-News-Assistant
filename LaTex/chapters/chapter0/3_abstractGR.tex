% Greek Abstract
\begin{center}
  \centering

  \vspace{0.5cm}
  \centering
  \textbf{\Large{Περίληψη}}
  \phantomsection
  \addcontentsline{toc}{section}{Περίληψη}

  \vspace{1cm}

\end{center}

Εκατοντάδες άρθρα ειδήσεων, και όχι μόνο, δημοσιεύονται καθημερινά στο μικρο-διαδίκτυο μας, καθιστώντας αδύνατη την ανάγνωση του μεγαλύτερου μέρους τους λόγω των γρήγορων ρυθμών της καθημερινότητας. Ταυτόχρονα, η σχετικότητα των άρθρων με την είδηση μειώνεται με την πάροδο του χρόνου, καθώς τα νέα αλλάζουν συνεχώς ακόμη και αν πρόκειται για την ίδια είδηση. Επομένως, είναι απαραίτητη η εποπτεία τους σε πραγματικό χρόνο. Η αυτοματοποιημένη ανάκτηση αυτής της πληροφορίας είναι αναγκαία και μπορεί να επιτευχθεί χρησιμοποιώντας τεχνικές επεξεργασίας και κατανόησης φυσικής γλώσσας.

Η παρούσα διπλωματική εργασία μελετά την ανάπτυξη ενός ειδησεογραφικού βοηθού πραγματικού χρόνου. Πιο συγκεκριμένα, ο βοηθός είναι υπεύθυνος για την αναζήτηση και την εύρεση απαντήσεων στις ερωτήσεις του χρήστη, τις οποίες αναγνωρίζει κατά τη συνομιλία με αυτόν και η αναζήτηση πραγματοποιείται μέσω ενός συστήματος ερώτησης-απάντησης (\emph{Question-Answering - QA}). Η χρήση του βοηθού καθιστά την υλοποίηση πιο ευέλικτη και φιλική προς το χρήστη. Η είσοδος του συστήματος ορίζεται ως οτιδήποτε εισάγει ο χρήστης στην επικοινωνία του με τον ψηφιακό βοηθό. Οι ειδήσεις που υποστηρίζονται αφορούν πολιτική, αθλητικά, τεχνολογία, ταινίες και ηλεκτρονικά παιχνίδια. Τα άρθρα παρέχονται από την προσομοίωση ενός εξωτερικού συστήματος και εισέρχονται σε έναν ταξινομητή για τον καθορισμό της κατηγορίας τους. Στη συνέχεια, αποθηκεύονται σε μία βάση δεδομένων στην οποία ανατρέχει το \emph{QA} σύστημα για την εξαγωγή της απάντησης. Το τελικό σύστημα αποτελείται από τον ταξινομητή, τον ψηφιακό βοηθό και το σύστημα \emph{QA}. Κάθε ένα από αυτά τα κομμάτια μπορεί να αντικατασταθεί και βελτιστοποιηθεί ξεχωριστά από τα υπόλοιπα δημιουργώντας έτσι ένα αρθρωτό σύστημα.

% Η παρούσα διπλωματική εργασία μελετά την ανάπτυξη ενός ειδησεογραφικού βοηθού πραγματικού χρόνου. Πιο συγκεκριμένα Η χρήση του βοηθού καθιστά την υλοποίηση πιο ευέλικτη και φιλική προς το χρήστη. Η είσοδος του συστήματος ορίζεται ως οτιδήποτε εισάγει ο χρήστης στην επικοινωνία του με τον ψηφιακό βοηθό. Ο βοηθός είναι υπεύθυνος για την αναζήτηση και την εύρεση απαντήσεων στις ερωτήσεις του χρήστη, τις οποίες αναγνωρίζει κατά την συνομιλία με αυτόν και η αναζήτηση πραγματοποιείται μέσω ενός συστήματος ερώτησης-απάντησης (\emph{Question-Answering - QA}). Οι ειδήσεις που υποστηρίζονται αφορούν πολιτική, αθλητικά, τεχνολογία, ταινίες και ηλεκτρονικά παιχνίδια. Τα άρθρα παρέχονται από την προσομοίωση ενός εξωτερικού συστήματος και εισέρχονται σε έναν ταξινομητή για τον καθορισμό της κατηγορίας τους. Στη συνέχεια αποθηκεύονται σε μία βάση δεδομένων στην οποία ανατρέχει το \emph{QA} σύστημα για την εξαγωγή της απάντησης. Το τελικό σύστημα αποτελείται από τον ταξινομητή, τον ψηφιακό βοηθό και το σύστημα \emph{QA}. Κάθε ένα από αυτά τα κομμάτια μπορεί να αντικατασταθεί και βελτιστοποιηθεί ξεχωριστά από τα υπόλοιπα δημιουργώντας έτσι ένα αρθρωτό σύστημα.

Για τη διασφάλιση της σωστής λειτουργίας του ψηφιακού βοηθού αλλά και τη συντήρηση του χρησιμοποιήθηκε βοηθητικό εργαλείο για την τροφοδοτούμενη από συνομιλίες ανάπτυξη του (\emph{Conversation-Driven Development}). Μέσω αυτού, πραγματοποιείται η αξιολόγηση του συνολικού συστήματος σε πραγματικές συνομιλίες οι οποίες στη συνέχεια αξιοποιούνται για τη βελτιστοποίηση τόσο του βοηθού όσο και των υπόλοιπων κομματιών του.

Για την επιλογή των μοντέλων μηχανικής μάθησης για επεξεργασία φυσικής γλώσσας πραγματοποιήθηκαν πειράματα με τα οποία αξιολογήθηκε η αποδοτικότητα τους σε συγκεκριμένα σύνολα δεδομένων. Με τον τρόπο αυτό, μελετώντας τα αποτελέσματα των πειραμάτων, επιλέχθηκαν οι βέλτιστες παράμετροι για τη λειτουργία του συνολικού συστήματος.

\begin{flushright}
  \vspace{1cm}
  Μιχαήλ Δοϊνάκης
  \\
  Τμήμα Ηλεκτρολόγων Μηχανικών $\&$ Μηχανικών Υπολογιστών,
  \\
  Αριστοτέλειο Πανεπιστήμιο Θεσσαλονίκης, Ελλάδα
  \\
  Ιούλιος 2022
\end{flushright}