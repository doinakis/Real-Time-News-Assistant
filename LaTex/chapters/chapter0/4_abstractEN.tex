%  English Abstract
{\fontfamily{cmr}\selectfont

\phantomsection
\addcontentsline{toc}{section}{Abstract}


\begin{center}
  \centering
  \textbf{\Large{Title}}
  \vspace{0.5cm}

  \textbf{\large{Real time news assistant}}

  \vspace{1cm}

  \centering
  \textbf{Abstract}
\end{center}

Hundreds of news articles are published daily on our micro-internet, making it \linebreak impossible to read much of them in the fast pace of everyday life. Moreover, the relevance of the articles to the news decays over time, as the news are constantly changing even if they are referring  on the same subject. It is therefore necessary to monitor them in real time. Automated retrieval of this information is necessary and can be achieved using natural language processing and understanding techniques.

The current thesis studies the development of a real-time news assistant. The assistant is responsible for searching and finding answers to the user's questions, which are identified during the conversation and the search is performed through a question-answering (QA) system. The use of the assistant makes the implementation more flexible and user-friendly. The input of the system is defined as anything the user inserts in his conversation with the digital assistant. The news supported are politics, sports, technology, movies and computer games. Articles are provided by the simulation of an external system and are fed into a classifier to determine their category. They are then stored in a database which is referenced by the QA system to extract the answer. The final system consists of the classifier, the digital assistant and the QA system. Each of these components can be replaced and optimized  separately, thus creating a modular system.

To ensure the proper functioning of the digital assistant and its maintenance, an auxiliary  tool for conversation-driven development (\emph{Conversation-Driven Development}) was used. Through this, the evaluation of the overall system in real conversations are carried out, and are used to optimize both the assistant and the other components of the system.

Experiments were conducted on specific datasets in order to evaluate the efficiency of several machine learning models for natural language processing. By evaluating the results of the experiments, the optimal hyper-parameters for the operation of the overall system were selected.

\begin{flushright}
  \vspace{2cm}
  Michail Doinakis
  \\
  Electrical \& Computer Engineering Department,
  \\
  Aristotle University of Thessaloniki, Greece
  \\
  July 2022
\end{flushright}
}