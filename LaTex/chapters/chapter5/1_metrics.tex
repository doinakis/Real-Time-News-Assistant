\section{Μετρικές Αξιολόγησης Συστημάτων}
\label{sec:metrics}
\subsection{Αξιολόγηση μοντέλων ταξινόμησης}
Μια ταξινόμηση ορίζεται ως αληθώς θετική (\emph{True Positive - TP}) όταν όταν ταξινομείται σωστά στη θετική κλάση, ως ψευδώς αρνητική (\emph{False Negative - FN}) όταν ταξινομείται σε κλάση διαφορετική της θετικής, ενώ ανήκει στη θετική, ως ψευδώς θετική (\emph{False Positive - FP}) όταν ταξινομείται στη θετική κλάση ενώ δεν ανήκει σε αυτή και ως αληθώς αρνητική (True Negative - TN) όταν δεν ανήκει στη θετική κλάση και δεν ταξινομείται σε αυτή. Με βάση αυτούς τους ορισμούς προκύπτουν οι μετρικές οι οποίες αξιοποιήθηκαν για την αξιολόγηση των μοντέλων ταξινόμησης.

\begin{itemize}
    \item \emph{Accuracy}: περιγράφει τον αριθμό των σωστών προβλέψεων σε σχέση με το συνολικό αριθμό των προβλέψεων, \autoref{eq:accuracy}
    \item \emph{Precision}: είναι ένα μέτρο για το πόσες από τις θετικές προβλέψεις είναι σωστές, \autoref{eq:precision}
    \item \emph{Recall}: είναι ο λόγος του αριθμού των θετικών προβλέψεων που ταξινομήθηκαν σωστά προς των συνολικό αριθμό των θετικών, \autoref{eq:recall}
    \item $F_1$ σκορ\footnote{Στους πίνακες που ακολουθούν παρουσιάζονται και οι μετρικές \emph{macro avg} και \emph{weighted avg} οι οποίες αποτελούν τον απλό και ζυγισμένο μέσο όρο του $F_1 Score$. Στο ζυγισμένο μέσο όρο λαμβάνεται υπόψιν και το \emph{support} κάθε κλάσης.} ($F_1 score$): είναι ένας συνδυασμός των \emph{precision} και \emph{recall}, \autoref{eq:f1-score}
    \item Υποστηριξη (\emph{Support}): δείχνει τον αριθμό των δειγμάτων κάθε κλάσης
\end{itemize}

\begin{align}
    &accuracy& &=& &\frac{\mathbf{TP} + \mathbf{TN}}{\mathbf{TP} + \mathbf{TN} + \mathbf{FP} + \mathbf{FN}} \label{eq:accuracy}\\
    &precision& &=& &\frac{\mathbf{TP}}{\mathbf{TP} + \mathbf{FP}}
    \label{eq:precision}\\
    &recall& &=& &\frac{\mathbf{TP}}{\mathbf{TP} + \mathbf{FN}}
    \label{eq:recall}\\
    &F_1 score& &=& &2 \cdot \frac{\mathbf{precision} \cdot \mathbf{recall}}{\mathbf{precision}+\mathbf{recall}}
    \label{eq:f1-score}
\end{align}

Σε προβλήματα ταξινόμησης όπου οι κλάσεις είναι παραπάνω των δύο, όπως και στη παρούσα περίπτωση όπου το άρθρο μπορεί να ταξινομηθεί σε μία από έξι πιθανές κλάσεις, τότε οι μετρικές που παρουσιάστηκαν παραπάνω εξάγονται για κάθε κλάση ορίζοντας διαδοχικά τη μία κλάση ως θετική και τις άλλες ως αρνητική.

\subsection{Αξιολόγηση μοντέλων \emph{Reader}}

Για την αξιολόγηση των μοντέλων των \emph{Reader} χρησιμοποιήθηκαν δύο μετρικές η απόλυτη ταύτιση (\emph{Exact Match, EM}) και μία παραλλαγή του \emph{$F_1 Score$} προσαρμοσμένη για συστήματα κατανόησης φυσικής γλώσσας. Το \emph{EM}, όπως μαρτυρά και η ονομασία του, αναφέρεται στη πλήρη ταύτιση της εξόδου του μοντέλου με την πραγματική απάντηση. Πιο συγκεκριμένα, είναι ο λόγος των σωστά προβλεπόμενων απαντήσεων από το μοντέλο, προς τις συνολικές. Γίνεται εύκολα αντιληπτό πως αυτή είναι μία αρκετά αυστηρή αξιολόγηση του μοντέλου καθώς η πρόβλεψη αυτού μπορεί να είναι πολύ κοντά στην πραγματική απάντηση, αλλά να διαφέρει για παράδειγμα κατά μία λέξη. Για το λόγο αυτό χρησιμοποιείται το \emph{$F_1 Score$}.

Ο υπολογισμός του \emph{$F_1 Score$} φαίνεται στην \autoref{eq:f1-score} όπου:
\begin{itemize}
    \item \textbf{\emph{precision}}: ο λόγος των κοινών λέξεων της προβλεπόμενης απάντησης με τη πραγματική, προς τον συνολικό αριθμό των λέξεων στην προβλεπόμενη απάντηση
    \item \textbf{\emph{recall}}: ο λόγος των κοινών λέξεων της προβλεπόμενης απάντησης με τη πραγματική, προς τον συνολικό αριθμό των λέξεων στην σωστή απάντηση
\end{itemize}