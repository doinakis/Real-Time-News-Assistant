\section{Σύνολο Δεδομένων Αξιολόγησης}
\label{sec:eval-dataset}
Εξαιτίας της έλλειψης συνόλων δεδομένων, πέρα από την ανάπτυξη μοντέλων μηχανικής μάθησης, καθίσταται δύσκολη και η αξιολόγηση των ελληνικών συστημάτων ειδικά αυτά που αφορούν συστήματα ερώτησης-απάντησης. Το \emph{XQuAD} \cite{xquad} είναι ένα διαγλωσσικό σύνολο δεδομένων ερώτησης απάντησης που αποτελείται από $240$ παραγράφους και $1190$ ερωτήσεις και είναι επαγγελματικά μεταφρασμένο για διάφορες γλώσσες ανάμεσα στις οποίες είναι και τα ελληνικά.

Για την αξιολόγηση των μοντέλων ταξινόμησης χρησιμοποιήθηκε μέρος του συνόλου των άρθρων που αποκτήθηκαν με τη μέθοδο που περιγράφεται στο \autoref{sec:soup}. Τα άρθρα λήφθηκαν από τρεις δημοφιλής ιστοσελίδες, μία για αθλητικές ειδήσεις, μία για πολιτικές ειδήσεις και μία για ειδήσεις τεχνολογίας, ταινιών και \emph{gaming}. Τα άρθρα της κατηγορίας άλλο (\emph{other}) λήφθηκαν με παρόμοιο τρόπο από το σύνολο δεδομένων της εφημερίδας Μακεδονία\footnote{\url{https://www.greek-language.gr/greekLang/modern_greek/tools/corpora/makedonia/content.html}}, από κατηγορίες που δεν επικάλυπταν τις παραπάνω. Για κάθε ένα από τα άρθρα που αποθηκεύονται στη βάση δεδομένων, εκτός από το περιεχόμενο αποθηκεύονται επίσης το όνομα του αρχείου, ο τίτλος του άρθρου, το \emph{ulr} της ιστοσελίδας από την οποία προήλθε το άρθρο, η κατηγορία του και η ημερομηνία έκδοσης του. Το σύνολο των ληφθέντων άρθρων και ένα παράδειγμα ενός άρθρου αποθηκευμένου στη βάση φαίνονται στους πίνακες \ref{tab:dataset} και \ref{tab:dataset-sample} αντίστοιχα.

\begin{table}[htb]
    \captionsetup{justification=centering}
    \begin{center}
        \caption{Σύνολο δεδομένων κάθε μίας από τις κατηγορίες των άρθρων}
        \begin{tabular}{ | c | c |}
            \hline
            \rowcolor{Gray}
            Κατηγορία & Αριθμός άρθρων\\
            αθλητικά & $5468$\\
            πολιτική & $1483$\\
            τεχνολογία & $1329$\\
            \emph{gaming} & $1273$\\
            ταινίες &  $1222$\\
            άλλο & $1427$\\
            \hline
            \hline
            \emph{Σύνολο} & $12202$\\
            \hline
        \end{tabular}
        \label{tab:dataset}
    \end{center}
\end{table}

\begin{table}[htb]
    \captionsetup{justification=centering}
    \begin{center}
        % \begin{tabular}{ |p{\columncolor[gray]{2.5cm}}|p{2cm}|p{2cm}|p{2cm}|p{2cm}|p{2.5cm}|}
        %     % \hline
        %     % \rowcolor{Gray}
        %     Περιεχόμενο & Όνομα αρχείου & Τίτλος & Url & Κατηγορία & Ημερομηνία έκδοσης\\
        %     Μαζί με το Mac Studio παρουσιάστηκε και το Studio Display, μία οθόνη με σώμα αλουμινίου ... & apple studio display .json & Studio Display: Η οθόνη της Apple έχει ένα Α13 Bionic και κάμερα iPhone & https:// url.com/ article0 & \emph{tech} & 2022-04-30\\
        %     \hline
        % \end{tabular}
        \caption{Παράδειγμα αποθηκευμένου άρθρου στη βάση δεδομένων}
        \begin{tabular}{|>{\columncolor[gray]{0.8}}c|m{9cm}|}
        \hline
            Περιεχόμενο & Μαζί με το Mac Studio παρουσιάστηκε και το Studio Display, μία οθόνη με σώμα αλουμινίου ...\\\hline
            Όνομα αρχείου & apple\_studio\_display.json\\\hline
            Τίτλος & Studio Display: Η οθόνη της Apple έχει ένα Α13 Bionic και κάμερα iPhone\\\hline
            Url & https://url.com/article0\\\hline
            Κατηγορία & \emph{tech}\\\hline
            Ημερομηνία έκδοσης & 2022-04-30\\\hline
        \end{tabular}
        \label{tab:dataset-sample}
    \end{center}
\end{table}