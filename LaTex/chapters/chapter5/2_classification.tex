\section{\emph{Classification}}
\label{sec:classification_exp}
Για την ταξινόμηση των άρθρων μελετήθηκαν ένα \emph{MLP} και μία \emph{fine-tuned} έκδοση του \emph{greek-bert}, όπως αναφέρθηκε στο \autoref{sec:classification}. Από τα αποτελέσματα που παρουσιάζονται στους πίνακες \ref{tab:classification-mlp} και \ref{tab:classification-bert}, παρατηρείται ότι και τα δύο μοντέλα έχουν αρκετά καλά αποτελέσματα για όλες τις κλάσεις ταξινόμησης. Παρατηρήθηκε ωστόσο ότι το \emph{fine-tuning} του \emph{greek bert} χρειάζεται αρκετά περισσότερο χρόνο για την εκπαίδευση του σε σχέση με το απλό \emph{MLP} που σχεδιάστηκε. Παρόλα αυτά, το τελικό μοντέλο που επιλέγεται είναι αυτό του \emph{greek BERT} διότι η τεχνολογία του με τα \emph{transformers} επιτρέπει και τη νοηματική "κατανόηση" της εισόδου σε αντίθεση με το \emph{MLP} το οποίο χρησιμοποιεί τη μέθοδο \emph{TF-IDF}. Έτσι, αν για παράδειγμα ερχόταν ως είσοδος στο σύστημα μία ταινία η οποία δεν βρισκόταν στα δεδομένα εκπαίδευσης το \emph{BERT} θα ήταν πιο πιθανό να προβλέψει ότι το άρθρο αφορά ταινία από τα συμφραζόμενα της εισόδου, κάτι το οποίο σίγουρα δεν υποστηρίζεται από το \emph{MLP}.


\begin{table}[!htb]
    \captionsetup{justification=centering}
    \begin{center}
        \caption{Αποτελέσματα ταξινόμησης με το μοντέλο \emph{MLP}}
        \begin{tabular}{ | c | c | c | c | c |}
            \hline
            \rowcolor{Gray}
            Κατηγορία & \emph{precision} & \emph{recall} & \emph{$F_1$ Score} & \emph{support}\\
            αθλητικά & $98.9$ & $99.8$ & $99.4$ & $656$\\
            πολιτική & $98.9$ & $97.2$ & $98.0$ & $178$\\
            τεχνολογία & $94.6$ & $98.1$ & $96.3$ & $160$\\
            \emph{gaming} & $98.7$ & $96.1$ & $97.4$ & $153$\\
            ταινίες & $99.3$ & $98.6$ & $99.0$ & $147$\\
            άλλο & $97.0$ & $94.7$ & $95.9$ & $171$\\
            \hline
            \hline
            \multicolumn{1}{| c |}{\emph{macro avg}} & $97.9$ & $97.4$ & $97.7$ & $1465$\\
            \hline
            \multicolumn{1}{| c |}{\emph{weighted avg}} & $98.2$ & $98.2$ & $98.2$ & $1465$\\
            \hline
            \hline
            \multicolumn{1}{| c |}{\emph{Accuracy}} & \multicolumn{4}{| c |}{$98.2$}\\
            \hline
        \end{tabular}
        \label{tab:classification-mlp}
    \end{center}
\end{table}

\begin{table}[!htb]
    \captionsetup{justification=centering}
    \begin{center}
        \caption{Αποτελέσματα ταξινόμησης με το μοντέλο \emph{Greek Bert}}
        \begin{tabular}{ | c | c | c | c | c |}
            \hline
            \rowcolor{Gray}
            Κατηγορία & \emph{precision} & \emph{recall} & \emph{$F_1$ Score} & \emph{support}\\
            αθλητικά & $99.8$ & $99.4$ & $99.6$ & $656$\\
            πολιτική & $100.0$ & $100.0$ & $100.0$ & $178$\\
            τεχνολογία & $95.2$ & $98.8$ & $96.9$ & $160$\\
            \emph{gaming} & $98.0$ & $96.7$ & $97.4$ & $153$\\
            ταινίες & $98.6$ & $97.3$ & $97.9$ & $147$\\
            άλλο & $98.8$ & $99.4$ & $99.1$ & $171$\\
            \hline
            \hline
            \multicolumn{1}{| c |}{\emph{macro avg}} & $98.4$ & $98.6$ & $98.5$ & $1465$\\
            \hline
            \multicolumn{1}{| c |}{\emph{weighted avg}} & $98.9$ & $98.9$ & $98.9$ & $1465$\\
            \hline
            \hline
            \multicolumn{1}{| c |}{\emph{Accuracy}} & \multicolumn{4}{| c |}{$98.4$}\\
            \hline
        \end{tabular}
        \label{tab:classification-bert}
    \end{center}
\end{table}

Για την ταξινόμηση των ερωτήσεων κατά την είσοδο τους στο \emph{QA} σύστημα, εξαιτίας της έλλειψης συνόλου δεδομένων ερωτήσεων, δοκιμάστηκε η χρήση του ίδιου μοντέλου που χρησιμοποιήθηκε για την ταξινόμηση των άρθρων. Η κύρια ιδέα είναι πως το περιεχόμενο των ερωτήσεων θα είναι παρόμοιο με αυτό των άρθρων, επομένως το μοντέλο θα είναι σε θέση να τις ταξινομήσει σωστά. Ωστόσο, όπως παρουσιάζεται και στον \autoref{tab:question-classification-bert}, αυτό δεν ισχύει για όλες τις κατηγορίες. Για το λόγο αυτό στη παρούσα υλοποίηση ο ταξινομητής ερωτήσεων παραλείπεται από το συνολικό σύστημα και η ερώτηση περνάει κατευθείαν στον \emph{Retriever}. Η αξιολόγηση έγινε σε σύνολο δεδομένων με ερωτήσεις που συλλέχτηκε από κοινό, αλλά το μέγεθος του συνόλου δεν επαρκούσε για την εκπαίδευση νέου μοντέλου.

\begin{table}[!htb]
    \captionsetup{justification=centering}
    \begin{center}
        \caption{Αποτελέσματα ταξινόμησης ερωτήσεων με το μοντέλο \emph{Greek Bert}}
        \begin{tabular}{ | c | c | c | c | c |}
            \hline
            \rowcolor{Gray}
            Κατηγορία & \emph{precision} & \emph{recall} & \emph{$F_1$ Score} & \emph{support}\\
            αθλητικά & $96.9$ & $81.6$ & $88.6$ & $38$\\
            πολιτική & $100.0$ & $18.4$ & $31.1$ & $38$\\
            τεχνολογία & $29.7$ & $100.0$ & $45.8$ & $38$\\
            \emph{gaming} & $92.9$ & $68.4$ & $78.8$ & $38$\\
            ταινίες & $100.0$ & $42.1$ & $59.3$ & $38$\\
            άλλο & $23.5$ & $10.5$ & $14.5$ & $38$\\
            \hline
            \hline
            \multicolumn{1}{| c |}{\emph{macro avg}} & $73.8$ & $53.5$ & $53.0$ & $228$\\
            \hline
            \multicolumn{1}{| c |}{\emph{weighted avg}} & $73.8$ & $53.5$ & $53.0$ & $228$\\
            \hline
            \hline
            \multicolumn{1}{| c |}{\emph{Accuracy}} & \multicolumn{4}{| c |}{$53.0$}\\
            \hline
        \end{tabular}
        \label{tab:question-classification-bert}
    \end{center}
\end{table}